\documentclass{article}

\title{Układy równań liniowych}
\author{Dominik Lau}

\usepackage{blindtext}
\usepackage{amsmath}
\usepackage[utf8]{inputenc}
\usepackage[polish]{babel}
\usepackage[T1]{fontenc}
\usepackage{listings}
\usepackage{color}
\usepackage{amssymb}
\usepackage{esvect}
\usepackage{graphicx}
\usepackage{hyperref}



\definecolor{dkgreen}{rgb}{0,0.6,0}
\definecolor{gray}{rgb}{0.5,0.5,0.5}
\definecolor{mauve}{rgb}{0.58,0,0.82}

\lstset{frame=tb,
  language=Python,
  aboveskip=3mm,
  belowskip=3mm,
  showstringspaces=false,
  columns=flexible,
  basicstyle={\small\ttfamily},
  numbers=none,
  numberstyle=\tiny\color{gray},
  keywordstyle=\color{blue},
  commentstyle=\color{dkgreen},
  stringstyle=\color{mauve},
  breaklines=true,
  breakatwhitespace=true,
  tabsize=3
}
\renewcommand\thesubsection{\Alph{subsection}}

\begin{document}
\maketitle
\section{Wstęp}
Celem projektu była implementacja i przeanalizowanie metod rozwiązywania układów równań liniowych
\begin{gather*}
	\boldsymbol{A}\boldsymbol{x} = \boldsymbol{b}
\end{gather*}
Rozważane metody to metoda faktoryzacji LU, metoda Jacobiego i metoda Gaussa-Seidla.  
Do implementacji wykorzystano język Python oraz bibliotekę matplotlib.
\section{Teoria}
\subsection*{Metoda faktoryzacji LU}
Jest to metoda bezpośredniego rozwiązywania układu równań.  Pierw rozbijamy macierz $\textbf{A}$
na dwie macierze trójkątne: dolną $\textbf{L}$ i górną $\textbf{U}$
\begin{gather*}
	\boldsymbol{A} = \boldsymbol{L}\boldsymbol{U}
\end{gather*}
następnie rozwiązujemy dwa układy równań dla macierzy trójkątnych
\begin{gather*}
	\boldsymbol{A}\boldsymbol{x} = \boldsymbol{b}\\
	 \boldsymbol{L} \boldsymbol{U}  \boldsymbol{x} =  \boldsymbol{b} \\
	 \boldsymbol{Ux} =  \boldsymbol{y}  \\\\
	 \boldsymbol{L}  \boldsymbol{y} =   \boldsymbol{b}
\end{gather*}
powyższe równanie rozwiązujemy dla $ \boldsymbol{y}$, następnie rozwiązujemy dla $\boldsymbol{x}$
korzystając z zależności
\begin{gather*}
	\boldsymbol{Ux} =  \boldsymbol{y}  \\\\
\end{gather*}
\subsection*{Metoda Jacobiego}
Jest to metoda iteracyjnego rozwiązywania układu równań. Pierw rozbijamy $\boldsymbol{A}$
na macierz $\boldsymbol{D}$ (diagonalną), $\boldsymbol{L}$ (trójkątną dolną) oraz $\boldsymbol{U}$ 
(trójkątną górną).
\begin{gather*}
	\boldsymbol{A} = \boldsymbol{D} + \boldsymbol{L} + \boldsymbol{U}
\end{gather*}
następnie iterujemy poniższe przybliżenie
\begin{gather*}
	\boldsymbol{x}^{(0)} = \vec{0} \\
	\boldsymbol{x}^{(k+1)} = \boldsymbol{D}^{-1}(\boldsymbol{b} - (
\boldsymbol{L} + \boldsymbol{U})\boldsymbol{x}^{(k)})
\end{gather*}
do momentu uzyskania żądanej dokładności.

\section{Analiza}
\subsection{Tworzenie testowego układu równań}

\section{Podsumowanie}
\section{Źródła}
\begin{itemize}
	\item \href{https://en.wikipedia.org/wiki/Jacobi_method}{Wikipedia-Jacobi}
\end{itemize}



\end{document}


