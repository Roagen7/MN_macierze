\documentclass{article}

\title{Układy równań liniowych}
\author{Dominik Lau}

\usepackage{blindtext}
\usepackage{amsmath}
\usepackage[utf8]{inputenc}
\usepackage[polish]{babel}
\usepackage[T1]{fontenc}
\usepackage{listings}
\usepackage{color}
\usepackage{amssymb}
\usepackage{esvect}
\usepackage{graphicx}
\usepackage{hyperref}



\definecolor{dkgreen}{rgb}{0,0.6,0}
\definecolor{gray}{rgb}{0.5,0.5,0.5}
\definecolor{mauve}{rgb}{0.58,0,0.82}

\lstset{frame=tb,
  language=Python,
  aboveskip=3mm,
  belowskip=3mm,
  showstringspaces=false,
  columns=flexible,
  basicstyle={\small\ttfamily},
  numbers=none,
  numberstyle=\tiny\color{gray},
  keywordstyle=\color{blue},
  commentstyle=\color{dkgreen},
  stringstyle=\color{mauve},
  breaklines=true,
  breakatwhitespace=true,
  tabsize=3
}
\renewcommand\thesubsection{\Alph{subsection}}

\begin{document}
\maketitle
\section{Wstęp}
Celem projektu była implementacja i przeanalizowanie metod rozwiązywania układów równań liniowych
\begin{gather*}
	\boldsymbol{A}\boldsymbol{x} = \boldsymbol{b}
\end{gather*}
Rozważane metody to metoda faktoryzacji LU, metoda Jacobiego i metoda Gaussa-Seidla.  
Do implementacji wykorzystano język Python oraz bibliotekę matplotlib.
\section{Teoria}
\subsection*{Metoda faktoryzacji LU}
Jest to metoda bezpośredniego rozwiązywania układu równań.  Pierw rozbijamy macierz $\textbf{A}$
na dwie macierze trójkątne: dolną $\textbf{L}$ i górną $\textbf{U}$
\begin{gather*}
	\boldsymbol{A} = \boldsymbol{L}\boldsymbol{U}
\end{gather*}
następnie rozwiązujemy dwa układy równań dla macierzy trójkątnych
\begin{gather*}
	\boldsymbol{A}\boldsymbol{x} = \boldsymbol{b}\\
	 \boldsymbol{L} \boldsymbol{U}  \boldsymbol{x} =  \boldsymbol{b} \\
	 \boldsymbol{Ux} =  \boldsymbol{y}  \\\\
	 \boldsymbol{L}  \boldsymbol{y} =   \boldsymbol{b}
\end{gather*}
powyższe równanie rozwiązujemy dla $ \boldsymbol{y}$, następnie rozwiązujemy dla $\boldsymbol{x}$
korzystając z zależności
\begin{gather*}
	\boldsymbol{Ux} =  \boldsymbol{y}  \\\\
\end{gather*}
\subsection*{Metoda Jacobiego}
Jest to metoda iteracyjnego rozwiązywania układu równań. Pierw rozbijamy $\boldsymbol{A}$
na macierz $\boldsymbol{D}$ (diagonalną), $\boldsymbol{L}$ (trójkątną dolną) oraz $\boldsymbol{U}$ 
(trójkątną górną).
\begin{gather*}
	\boldsymbol{A} = \boldsymbol{D} + \boldsymbol{L} + \boldsymbol{U}
\end{gather*}
następnie iterujemy poniższe przybliżenie
\begin{gather*}
	\boldsymbol{x}^{(0)} = \vec{0} \\
	\boldsymbol{x}^{(k+1)} = \boldsymbol{D}^{-1}(\boldsymbol{b} - (
\boldsymbol{L} + \boldsymbol{U})\boldsymbol{x}^{(k)})
\end{gather*}
do momentu uzyskania żądanej dokładności.
\subsection*{Metoda Gaussa-Seidla}
Jest to metoda bazująca na metodzie Jacobiego, też rozbijamy macierz $\boldsymbol{A}$ na 
$\boldsymbol{D}, \boldsymbol{L}, \boldsymbol{U}$. Zmienia się postać iteracji
\begin{gather*}
	\boldsymbol{x}^{(0)} = \vec{0} \\
	\boldsymbol{x}^{(k+1)} = (\boldsymbol{D} + \boldsymbol{L})^{-1}
(\boldsymbol{b} - \boldsymbol{U}\boldsymbol{x}^{(k)})
\end{gather*}

\subsection*{Norma residuum}
Do oceny ilościowej dokładności metod iteracyjnych liczy się normę wektora residuum.
Wektor residuum
\begin{gather*}
	\boldsymbol{res} = \boldsymbol{A}\boldsymbol{x} - \boldsymbol{b}
\end{gather*}
W projekcie wykorzystano zwykłą normę euklidesową
\begin{gather*}
	||\boldsymbol{x}|| = \sqrt{\Sigma_{x \in \boldsymbol{x}} x^2}
\end{gather*}
\section{Analiza}
\subsection{Tworzenie testowego układu równań}
W testowanym układzie równań
\begin{gather*}
	\boldsymbol{A} = \begin{pmatrix}
		a_1 & a_2 & a_3 & 0 & 0 & ...  & 0\\
		a_2 & a_1 & a_2 & a_3 & 0 & ...  & 0\\
		a_3 & a_2 & a_1 & a_2 & a_3 & ...  & 0\\
		...     & ...     &  ...    & ...     & ...     & ...  & ...\\
		0 & 0 & 0 & 0 & 0 & ...  & a_1\\
	\end{pmatrix} \\\\
	\boldsymbol{b} = \begin{pmatrix}
		sin(8 \cdot 1) \\
		sin(8 \cdot 2) \\
		... \\
		sin(8 \cdot n)
	\end{pmatrix}
\end{gather*}
gdzie $a_1=11, a_2 = a_3 = -1$ (nr indeksu 188697).
Dla tak dobranych parametrów macierz $\boldsymbol{A}$ jest przekątniowo dominująca
co gwarantuje zbieganie się algorytmów iteracyjnych (\textbf{kryterium silnej dominacji w wierszach}).
\subsection{Działanie metody Jacobiego i Gaussa-Seidla}
\begin{center}
\begin{tabular}{ c | c c c}
 metoda & czas [ms] & iteracje & $norm(res)$ \\ 
\hline
 Jacobiego & 2947 & 17 & $6,83 \cdot 10^{-10}$\\  
 Gaussa-Seidla & 2034 & 13 & $4,17 \cdot 10^{-10} $  
\end{tabular}
\end{center}
Warto zauważyć, że metoda Gaussa-Seidla zbiega się w tym przypadku w mniejszej ilości iteracji
i osiąga lepszy czas.
\subsection{Metody iteracyjne na zmienionym układzie równań}
Zamieniamy teraz elementy leżące na głównej przekątnej macierzy $\boldsymbol{A}$ na 3.
Macierz nie jest już przekątniowo dominująca zatem nie mamy gwarancji zbieżności metod iteracyjnych.
Rzeczywiście, metody nie zbiegają się. 

\subsection{Działanie faktoryzacji LU dla zmienionego układu równań}
\subsection{Zależności działania metod od czasu}
\section{Podsumowanie}
\section{Źródła}
\begin{itemize}
	\item \href{https://en.wikipedia.org/wiki/Jacobi_method}{Wikipedia-Jacobi}
	\item \href{https://pl.wikipedia.org/wiki/Macierz_przek%C4%85tniowo_dominuj%C4%85ca}{Macierz-
przekatniowo-dominujaca}
\end{itemize}



\end{document}


